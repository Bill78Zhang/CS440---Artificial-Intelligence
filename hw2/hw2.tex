\documentclass[titlepage]{article}
\usepackage{amsmath}

\title{CS 440 - Homework 2}
\author{Matthew Liu}
\date{}

\begin{document}
\maketitle{}

\noindent \textbf{1. } Suppose we could destroy any block in the otherwise traditional blocks world. The destruction is gentle with any supported blocks settling down on the support under the destroyed block.\\

\textbf{a) } Operators defined below

\begin{quote}
	DestroyBlkUnderBlk(?x)
	\begin{itemize}
		\item Preconditions: Blk(?x), Blk(?y), On(?y,?x), On(?x,?z)
		\item Delete: Blk(?x), On(?y,?x), On(?x,?z)
		\item Add: On(?y, ?z)
	\end{itemize}

	DestroyClrBlkOnTable(?x)
	\begin{itemize}
		\item Preconditions: Blk(?x), Clr(?x), On(?x,?y), Tbl(?y)
		\item Delete: Blk(?x), Clr(?x), On(?x,?y)
		\item Add: Nothing
	\end{itemize}

	DestroyClrBlkOnBlock(?x)
	\begin{itemize}
		\item Preconditions: Blk(?x), Clr(?x), On(?x,?y), Blk(?y)
		\item Delete: Blk(?x), Clr(?x), On(?x,?y)
		\item Add: Clr(?y)
	\end{itemize}
\end{quote}

\textbf{b) } Examples listed below

\begin{quote}
	Example with Destroy: Moving a block on a block on the table to the table. Using destroy would cost of $2$, whereas MoveToTable would cost $3$.
	\begin{itemize}
		\item Initial State: Blk(?x), Blk(?y), Tbl(?z), On(?x,?y), On(?y,?z)
		\item Actions: Destroy(?y)
		\item Goal State: Blk(?x), Tbl(?z), On(?x,?z)
	\end{itemize}

	Example without Destroy: Moving a block on the top of a stack of 3 blocks to the table. Using destroy would cost of $3x2=6$, whereas MoveToTable just costs $3$.
	\begin{itemize}
		\item Initial State: Blk(?a), Blk(?b), Blk(?c), Blk(?d), Tbl(?e), On(?a,?b), On(?b,?c), On(?c,?d), On(?d,?e)
		\item Actions: MoveToTable(?a)
		\item Goal State: Blk(?a), Blk(?b), Blk(?c), Blk(?d), Tbl(?e), On(?a,?e)
	\end{itemize}
\end{quote}

\newpage

\textbf{c) } Examples listed below

\begin{quote}
	Example with Destroy: Moving a block with 2 blocks on top and the block itself is on a block on the table to the table. Using destroy would evoke a cost of $4$, whereas MoveToTable then MoveToBlock would cost $3+2=5$.
	\begin{itemize}
		\item Initial State: Blk(?a), Blk(?b), Blk(?c), Blk(?d), Tbl(?e), On(?a,?b), On(?b,?c), On(?c,?d), On(?d,?e)
		\item Actions: Destroy(?d)
		\item Goal State: Blk(?a), Blk(?b), Blk(?c), Tbl(?e), On(?c,?e), On(?b, ?a)
	\end{itemize}

	Example without Destroy: Moving a block on a block on the table to the table. Using destroy would cost of $4$, whereas MoveToTable would cost $3$.
	\begin{itemize}
		\item Initial State: Blk(?x), Blk(?y), Tbl(?z), On(?x,?y), On(?y,?z)
		\item Actions: MoveToTable(?y)
		\item Goal State: Blk(?x), Tbl(?z), On(?x,?z)
	\end{itemize}
\end{quote}

\noindent \textbf{2. } Give the most natural English sentence you can for the following first-order logic sentences. If it is ambiguous, give all meanings. If it is not possible to translate into English, explain why.\\

\textbf{a) } Every human that drinks tea owns a cup.\\

\textbf{b) } Ambigious as the variable y is not defined, thus making it impossible to translate. It could be a human, in which case the statement is saying that for all people who can't shave themselves there exists a barber who shaves them and if there exists a barber who shaves a person, then that person cannot shave themselves.\\

\textbf{c) } Similar to b, y could be a human, in which case for all people who cannot shave themselves there exists a barber who will shave them, and the barber shaves himself or herself.\\

\newpage

\noindent \textbf{3. } Give a first-order logic sentence that best captures the meaning of the following English sentences. If the meaning of a predicate is not obvious, be sure to explain your intended semantics.\\

\textbf{a) } Not everyone likes onions

\centerline{$\exists x(\text{Human}(x) \wedge \forall y(\text{Onion}(y) \wedge \neg\text{Like}(x, y)))$}
\centerline{}

\textbf{b) } Some cats would do anything to cause trouble

\centerline{$\exists x(\text{Cat}(x) \wedge \forall y(\text{Action}(y) \Rightarrow \exists z(\text{Trouble}(z) \wedge \text{Cause}(z))))$}
\centerline{}

\textbf{c) } The road to hell is paved with good intentions

\centerline{$\exists x(\text{RoadToHell}(x) \wedge \exists y(\text{GoodIntentions}(y) \Rightarrow \text{Paved}(x, y)))$}
\centerline{}

\noindent \textbf{4. } Give the most general unifier and a corresponding unification instance for each set of expressions below. If unification is not possible, write ``No Unification'' and briefly explain why.\\

\textbf{a) } No Unification. There are too many circular dependencies, always leading to the repeat definition of a variable (as in a variable shows up on the left side of equal multiple times).\\

\textbf{b) } Unification below

\begin{quote}	
	\begin{itemize}
		\item MGU: $\{?z=?y, ?y=?w, ?w=?v, ?v=?x, ?u=?x, ?x=\text{Bob}\}$
		\item Unification Instance: $\text{Parents}(Bob, Bob, Bob, Bob)$
	\end{itemize}
\end{quote}

\textbf{c) } No Unification, Same problem as a, but one step further. Lets assume for one second that the MGU $\{?x=\text{Sister}(?u), \text{Sister}(?v)=?x, \text{Sister}(?u)=\text{Sister}(?v), ?v=?u Son(?v)=Son(?r), ?v=?r, ?r=\text{John}\}$, This leads to part of a unification instance $(?x=\text{Sister-of}(\text{John}))$. This however is impossible, as the questions notes that John has no sisters at all\\

\textbf{d) } No Unification. Dogs and Like are different functions, hence they can not be equated.

\end{document}